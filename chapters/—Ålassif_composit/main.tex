\documentclass[a4paper,12pt]{article}

\usepackage[12pt]{extsizes}
%%% Работа с русским языком
\usepackage{graphicx} % Для вставки рисунков
\usepackage{cmap} % поиск в PDF
\usepackage{mathtext} % русские буквы в фомулах
\usepackage[T2A]{fontenc} % кодировка
\usepackage[utf8]{inputenc} % кодировка исходного текста
\usepackage[english,russian]{babel}
\usepackage{setspace}
\usepackage[DIV=14]{typearea}
\usepackage[margin=1in]{geometry}
\usepackage{pgffor}

\usepackage{amsmath,amsfonts,amssymb,amsthm,mathtools} % AMS
\usepackage{icomma}

% поля:
%\usepackage[left=0.5cm, right=1.5cm, vmargin=2.5cm]{geometry}



%% Перенос знаков в формулах (по Львовскому)
\newcommand*{\hm}[1]{#1\nobreak\discretionary{}
{\hbox{$\mathsurround=0pt #1$}}{}}

%%% Работа с картинками

\usepackage{float}
\DeclareGraphicsExtensions{.pdf,.png,.jpg}
\setlength\fboxsep{3pt} % Отступ рамки \fbox{} от рисунка
\setlength\fboxrule{1pt} % Толщина линий рамки \fbox{}
\usepackage{wrapfig} % Обтекание рисунков текстом
\usepackage{indentfirst} % Следующая после секции строка с красной строки
\usepackage{misccorr} % в заголовках появляется точка, но при ссылке на них ее нет

%%% Работа с таблицами
\usepackage{array,tabularx,tabulary,booktabs} % Дополнительная работа с таблицами
\usepackage{longtable} %  Длинные таблицы
\usepackage{multirow} % Слияние строк в таблице

\usepackage{extsizes} % Возможность сделать 14-й шрифт

\usepackage{lastpage} % Узнать, сколько всего страниц в документе.

\usepackage{soul} % Модификаторы начертания

% в преамбуле
\usepackage[labelsep=space]{caption}

\addto\captionsrussian{%
\renewcommand{\figurename}{\CYRR\cyri\cyrs\cyru\cyrn\cyro\cyrk }%
}

\RequirePackage{caption}
\DeclareCaptionLabelSeparator{defffis}{ - }
\captionsetup{justification=centering,labelsep=defffis}

%\graphicspath{{pictures/}}


\usepackage{hyperref}
\usepackage[usenames,dvipsnames,svgnames,table,rgb]{xcolor}
\hypersetup{ % Гиперссылки
unicode=true, % русские буквы в раздела PDF
pdftitle={Заголовок}, % Заголовок
pdfauthor={Автор}, % Автор
pdfsubject={Тема}, % Тема
pdfcreator={Создатель}, % Создатель
pdfproducer={Производитель}, % Производитель
pdfkeywords={keyword1} {key2} {key3}, % Ключевые слова
colorlinks=true, % false: ссылки в рамках; true: цветные ссылки
linkcolor=red, % внутренние ссылки
citecolor=green, % на библиографию
filecolor=magenta, % на файлы
urlcolor=cyan % на URL
}


\usepackage{multicol} % Несколько колонок

\usepackage{hyperref} 
\usepackage[usenames,dvipsnames,svgnames,table,rgb]{xcolor} 
\hypersetup{ % Гиперссылки 
unicode=true, % русские буквы в раздела PDF 
pdftitle={Заголовок}, % Заголовок 
pdfauthor={Автор}, % Автор 
pdfsubject={Тема}, % Тема 
pdfcreator={Создатель}, % Создатель 
pdfproducer={Производитель}, % Производитель 
pdfkeywords={keyword1} {key2} {key3}, % Ключевые слова 
colorlinks=true, % false: ссылки в рамках; true: цветные ссылки 
linkcolor=black, % внутренние ссылки 
citecolor=green, % на библиографию 
filecolor=magenta, % на файлы 
urlcolor=cyan % на URL 
} 

\begin{document} 
\thispagestyle{empty} 
\begin{center} 
\textit{Федеральное государственное автономное учреждение \\ 
высшего образования} 
\vspace{0.5ex} 

\textbf{МОСКОВСКИЙ ФИЗИКО-ТЕХНИЧЕСКИЙ ИНСТИТУТ \\ (НАЦИОНАЛЬНЫЙ ИССЛЕДОВАТЕЛЬСКИЙ УНИВЕРСИТЕТ)} 
\end{center} 
\vspace{13ex} 
\begin{flushright} 
\noindent 
\textit{Емекеева Дарья} 
\\ 
\textit{студентка 1-го курса аспирантуры ФБМФ \\ группа А06-404} 
\end{flushright} 
\begin{center} 
\vspace{13ex}
 
\vspace{1ex} 
\textbf{\textit{«Нелинейные монотонные композиции»}} 

\vfill 
г. Долгопрудный 

\today 
\end{center}

\newpage 

\section{Основные обозначения} 
 \textbf{Определение 1}
 
 Произвольная линейная корректирующая операция $F(b_1,... , b_T ) = \alpha_1 b_1 + · · · + \alpha_T b_T$ с неотрицательными коэффициентами $\alpha_t$ является монотонным отображением из $R_T$ в Y и называется \textit{монотонной корректирующей операцией}.\\
 
 Искомый алгоритм $a$ имеет вид $a(x) = F(b_1(x), . . . , b_T (x))$
 
\subsection*{Лемма о проведении монотонной функции через заданные точки.}
 
 \textbf{Определение 2}
 
 Пусть U, V -- произвольные частично уторядоченные множества. Совокупность пар $(u_i, v_i)_{i=1}^l$ из $U\times V$ назвыается \textit{монотонной}, если из $u_i\leq u_k$ следует $v_i\leq v_k$ для всех $j,k = 1,...,l.$\\
 
 \textbf{Лемма 1}
 
Пусть  $U, V$ -- произвольные частично упорядоченные множества. Монотонная функция $f : U \longrightarrow V$ такая, что $f(u_i) = v_i$ для всех $i = 1, . . . , l$, существует
тогда и только тогда, когда совокупность $(u_i
, v_i)^l_{i=1}$ монотонна.

\textbf{Доказательство.}

Необходимость вытекает из определения монотонной функции: если $f$ монотонна, то совокупность $(u_i,f(u_i))^l_{i=1}$ монотонна.

Докажем достаточность. Предполагая, что совокупность $(u_i,f(u_i))^l_{i=1}$ монотонна,
построим функцию $f$ в классе кусочно-постоянных функций. Определим для произвольного $u \in U $ множество индексов $I(u) = {k : u_i \leq u}$ и положим

\begin{equation*}
    f(u) =
    \begin{cases}
        \min\limits_{i=1,...,l}v_i, \text{ если } I(u) = \varnothing; \\
        \max\limits_{i\in I(u)}v_i, \text{ если } I(u) \in \varnothing .
    \end{cases}
\end{equation*}


Докажем, что функция f монотонна. Для произвольных $u$ и $u\prime $ из $u_i \leq u$ следует $I(u)\subseteq I( u\prime)$, значит $f(u)\leq f(u\prime )$.
Докажем, что $f(u_i) = v_i$ для всех $i = 1, . . . , l$. Множество $I(u_i)$ не пусто, так
как $i \in I(u_i)$. Для любого $k \in I(u_i)$ в силу монотонности $(u_i, v_i)^l_{i=1}$ из $u_k \leq u_i$ следует $v_k \leq v_i$. Но тогда $max_{k\in I(u_i)} v_k$ достигается при $k = i$, откуда следует $f(u_i) = v_i$.$\blacksquare$\\




Предложенный способ построения функции $f$ мало пригоден для практических нужд.
 В регрессионных задачах желательно, чтобы функция $f$ была гладкой или хотя бы непрерывной,здесь  $f$ кусочно-постоянна. В задачах классификации  -- чтобы разделяющая поверхность проходила как можно дальше от точек выборки, здесь разделяющая полоса целиком относится к классу 0.


\section{Оптимизация базовых алгоритмов}

Обозначим через $u_i$ вектор значений базовых алгоритмов на объекте $x_i$, через $f_i$ — значение, выданное алгоритмом a(x) на объекте $x_i$:
\[u_i = (b_1(x_i), . . . , b_t(x_i));\]
\[f_i = a(x_i) = F(b_1(x_i), . . . , b_t(x_i)) = F(u_i); \; i = 1, . . . , l.\]
В новых обозначениях условие корректности алгоритма $a(x_i) = y_i$ примет вид
\[F(u_i) = y_i
, \; i = 1, . . . , l. 
\]\\

\textbf{Определение 3} 

Пусть $V$ — произвольное упорядоченное множество. Пара индексов $(j, k)$
называется дефектной для функции $b : X \rightarrow V$ , если $y_j < y_k$ и $b(x_j ) > b(x_k)$. Дефектом функции $b(x)$ называется множество всех её дефектных пар:
\[D(b) = {(j, k): y_j < y_k \wedge b(x_j ) > b(x_k)} .\]


Следовательно любой монотонный оператор, а следовательно и алгоритм $a(x) = F(b1(x), . . . , bt(x)) $ допустят ошибку в точке $x_j$ или $x_k$ при дефектной паре $(j, k)$.
\\

\textbf{Определение 4} 

Совокупным дефектом операторов $b_1, . . . , b_t$ называется множество
$D_T (b_1, . . . , b_t) = D(b_1) \cap ... \cap D(b_t) = {(j, k): y_j < y_k \wedge u_j > u_k}.$
Будем также пользоваться сокращённым обозначением $D_t = D_t(b_1, . . . , b_t).$
\\
Для любой пары $(j, k)$ из совокупного дефекта и любой монотонной функции $F$ алгоритм $a(x) = F(b_1(x), . . . , b_t(x)) $ допустит ошибку хотя бы на одном из двух объектов $x_j$ или $x_k$.\\

\textbf{
Теорема 1} 

 Монотонная функция $F : R^t \rightarrow Y $, удовлетворяющая условию корректности существует тогда и только тогда, когда совокупный дефект операторов $b_1, . . . , b_t$ пуст. При этом дефект алгоритма $a(x) = F(b_1(x), . . . , b_t(x)) $ также пуст.


\textbf{Доказательство.}

Справедлива следующая цепочка равносильных утверждений:
а) совокупный дефект пуст: $D_t(b_1, . . . , b_t) = \varnothing$;
б) для любых $j, k$ не выполняется $(u_k \leq u_j ) \wedge (y_j < y_k)$ (согласно Опр. 1.6);
в) для любых $j, k$ справедлива импликация $(u_k \leq u_j ) \rightarrow (y_k \leq y_j )$;
г) совокупность пар $(u_i, y_i)^l_{i=1}$ монотонна (согласно Опр. 1.4);
д) существует монотонная функция $F : R_t \rightarrow Y$ такая, что $F(u_i) = y_i$ для всех
$i = 1, . . . , l$ (согласно Лемме 1.5).
Из утверждения д) следует, что условия $y_j < y_k$ и $F(u_j ) > F(u_k)$ не могут
выполняться одновременно, значит, дефект $D(F(b_1, . . . , b_t))$ также пуст. $\blacksquare$\\


Таким образом, для корректной работы алгоритма необходимо построить операторы $ b_1, . . . , b_t$
, совокупный дефект которых пуст. А добавление такого оператора $b$, что 
\begin{equation}\label{iter}
b_t (x_j) < b_t(x_k) (j, k) \in D_{t-1}
\end{equation}
приводит к устранению дефектной пары.\\

\textbf{Теорема 2 (о сходимости)}. 

Пусть на первом шаге построен оператор $b_1$, и семейство операторов $B$ выбрано так, что для любой подвыборки $X^{2m}$
длины $2m, m > 1$, найдётся оператор $b\in B$ и монотонная корректирующая операция $F$, удовлетворяющие системе ограничений
\[
F(b(x_i)) = y_i
, x_i \in X^{2m}.\]
Тогда итерационный процесс, аналогичный последовательному построению смеси алгоритмов, приводит к построению корректной композиции
$a = F(b_1, . . . , b_T )$ за конечное число шагов $T \leq [D(b_1)/m]+ 1$.


\textbf{Доказательство.}

Рассмотрим t-й шаг, t > 2, итерационного процесса . Если $\vert D_{t-1}\vert > m$,
то выберем некоторое m-элементное подмножество совокупного дефекта $\Delta_{t-1} \subseteq D_{t-1}$.
Если $\vert D_{t-1}\vert \leq m$, то положим $\Delta{t-1} = D_{t-1}$. Рассмотрим подмножество объектов
выборки, образующих всевозможные дефектные пары из $\Delta{t-1}$:
\[ U = \lbrace x_i \in X^l \exists k : (k, i) \in \Delta_{t-1} \; \text{или} \; (i, k) \in \Delta_{t-1} \rbrace\].

Очевидно, мощность $U$ не превышает 2m. Согласно условию теоремы существует оператор $b_t \in B$ и монотонная корректирующая операция $F$, удовлетворяющие системе
ограничений $F(b_t(x_i)) = y_i$ при всех $x_i \in U$. Но тогда для оператора $b_t$ выполняется
также система ограничений
\[ b_t(x_j) < b_t(x_k) \; \, (j, k) \in \Delta_{t-1}\]


Докажем это от противного: пусть $b_t(x_k) \leq b_t(x_j )$, тогда $F(b_t(x_k)) \leq F(b_t(x_j ))$, следовательно, $y_k \leq y_j$
, что противоречит условию $y_j < y_k$, входящему в определение
дефектной пары $(j, k)$.
Если выбирать операторы $b_t , \; t = 2, 3, . . .$ указанным способом, то мощность
совокупного дефекта будет уменьшаться, как минимум, на m на каждом шаге итерационного процесса: $\vert D_t \vert \leq \vert D_{t-1}\vert -m$. Для полного устранения дефекта потребуется не более $\lceil D(b_1)/m\rceil$
операторов, не считая $b_1$. $\blacksquare$
\\

Процесс последовательного построения базовых алгоритмов организуется
по принципу \textit{наискорейшего устранения дефекта}.
Если специальным образом задать веса
обучающих объектов, мощность совокупного дефекта можно оценить сверху
обычным функционалом числа ошибок.

Следующая теорема показывает, что в случае классификации вес i-го объекта
можно положить равным числу пар из совокупного дефекта $D_{t-1}$, в которых участвует данный объект.\\

\textbf{Теорема 3} 

Если $Y = {0, 1}$ и функция потерь имеет вид $\overset{\sim}{L}(b, y) =[[b > 0]\neq y]$ , то справедлива оценка \[\vert D_t(b_1, . . . , b_t)\vert \leq \operatornamewithlimits{\sum}^l_{i=1} w_i \overset{\sim}{L} (b_t(x_i), y_i); \]
\[w_i = \vert D(i)\vert ;\]
\[D(i) = \lbrace x_k \in X^l \vert (k, i) \in D_{t-1} \text{ или } (i, k) \in D_{t-1} \rbrace; \; i = 1, . . . , l.\]

\textbf{Доказательство.}

Обозначим через $\beta_i = b_t(x_i)$ значение t-го оператора на i-м обучающем объекте, через $L_i = \overset{\sim}{L}(\beta_i, y_i) = [\beta_i > 0] \neq y_i$

— значение функции потерь, равное 1, если
оператор $b_t$ допускает ошибку на i-м объекте, 0 в противном случае.
Для любых действительных $\beta_j, \beta_k$ справедливо неравенство
\[[\beta_j > \beta_k] \leq [\beta_j > 0] + [\beta_k \leq 0]\].



Если $y_j < y_k$, то из двухэлементности множества Y следует:
\[y_j = 0; \; [\beta_j > 0] =[\beta_j > 0] \neq 0 = [\beta_j > 0] \neq y_j = L_j;\]
\[y_k = 1; \; [\beta_k \leq 0] =
[\beta_k > 0] \neq 1=[\beta_k > 0] \neq y_k= L_k.\]

Используя представление $D_t = D_{t-1} \bigcap D(b_t)$, распишем мощность совокупного
дефекта в виде суммы по парам объектов и применим оценку $[\beta_j > \beta_k] \in L_j + L_k$:
\[\vert D_t \vert = \operatornamewithlimits{\sum}_{(j,k)\in D_{t-1}} [(j, k) \in D(b_t)]=
\operatornamewithlimits{\sum}_{(j,k)\in D_{t-1}}  [\beta_j > \beta_k] \leq 
\operatornamewithlimits{\sum}_{(j,k)\in D_{t-1}}(L_j + L_k) =\]
\[ = \operatornamewithlimits{\sum}^l_{\operatornamewithlimits{j=1}\limits_{y_j=0}} L_j
\operatornamewithlimits{\sum}^l_{k=1}[(j, k) \in D_{t-1}] +
\operatornamewithlimits{\sum}^l_{\operatornamewithlimits{k=1}\limits_{y_k=1}} L_k
\operatornamewithlimits{\sum}^l_{j=1}[(j, k) \in D_{t-1}]=\]
\[=\operatornamewithlimits{\sum}^l_{i\in 1} L_i \operatornamewithlimits{\sum}^l_{k=1} [(k, i) \in D_{t-1}\text{ или }(i, k)\in D_{t-1}=\operatornamewithlimits{\sum}^l_{i\in 1} L_i \vert D(i)\vert.\] $\blacksquare$\\





%\begin{figure}[h!]
%	\centering
%	\includegraphics[height=0.3\textheight]{signal}
%	\caption{Зависимость тока от времени для GaAs в диоде Ганна}
%\end{figure}


\section{Задачи}
\textbf{\emph{Задача 1}}

Имеется алгоритм $a(x)=F(b_1(x)...,b_t(x))$, где $F$ монотонная корректирующая операция. 
Если $\forall i; \; i = 1,...,t; \; b_i\in B$, функция $a \in A$, является ли верным утверждение, что $A=B$?

\textit{Решение}

Класс $B$ может накладывать определенные ограничения на функции, которые он содержит (например, линейность, непрерывность, ограниченность). Монотонная операция $F$ может изменить эти ограничения.

 Монотонная операция F может расширить или сузить множество функций, которые могут быть получены. Eсли $F$ – нелинейная функция (например, $F(x_1, ..., x_t) = max(x_1, ..., x_t))$, то класс $A$ будет содержать нелинейные функции (даже если все $b_i$ линейны), которые не принадлежат классу $B$. В этом случае $A \neq B$ а именно $A \supset B $.\\

\textbf{\emph{Задача 2}}

Напишите пример монотонной корректирующей операции $F$, такой что в предыдущей задаче достигнется развенство классов $A$ и $B$.


\textit{Решение}

Пусть класс $B$ содержит только линейные функции вида $b_i(x) = k_i x + c_i$. Если $F(x_1, x_2) = x_1 + x_2$, то класс $A$ также будет содержать только линейные функции. В этом случае $A = B$.\\

\textbf{\emph{Задача 3}}

Рассмотрим класс функций $F$, представляющих собой композиции монотонно неубывающих функций. Пусть $b_i: \mathbb{R} \rightarrow \mathbb{R}$ для $i = 1, ..., t$ — монотонно неубывающие функции, и $g_i: \mathbb{R}^n \rightarrow \mathbb{R}$ для $i = 1, ..., t$ — также монотонно неубывающая функция. Тогда функция $F(x_1, ..., x_n) = g(f_1(x_1), ..., f_n(x_n))$ принадлежит классу $F$.

Верно ли следуюшее утверждение? Если функция $h(x_1, x_2) \in F$ удовлетворяет условиям:

\begin{itemize}
\item $h(0, 0) = 0$
\item $h(1, 1) = 1$
\item $h(x, y) = h(y, x)$
\item Для любых $x_1, x_2 \geq 0$ выполняется $h(x_1, x_2) \geq max(x_1, x_2)$,

\end{itemize}


то функция $h(x, y)$ тождественно равна $max(x, y)$ при $x, y \in [0, 1]$.\\

\textit{Решение}

Докажем от противного.

Пусть $\exists h(x_1, x_2) \in F, \; h(x_1, x_2) \neq max(x_1, x_2)$ на [0, 1], удовлетворяющая условиям 1-4.  Значит существует хотя бы одна пара $(x_0, y_0) \in [0, 1] \; : \; h(x_0, y_0)> max(x_0, y_0)$.

Предположим $x_0 \geq y_0$. Тогда $h(x_0, y_0) > x_0$.

Так как $h \in F$,  она является композицией монотонно неубывающих функций.  Это означает, что если $x_1 \geq x_2$ и $y_1 \geq y_2$, то $h(x_1, y_1) \geq h(x_2, y_2)$.

Рассмотрим точку $(x_0, x_0)$.  По свойству 3,$ h(x_0, y_0) = h(y_0, x_0)$.  По условию 4, $h(x_0, x_0) \geq x_0$.  Но так как $h(x_0, y_0) > x_0$ и $y_0 \neq x_0$,  из монотонности $h$ следует   $h(x_0, x_0) \geq h(x_0, y_0) > x_0$.

Теперь рассмотрим случай, когда $x_0 = 1$ и $y_0 < 1$.  Тогда $h(1, y_0) > 1$  (поскольку $h(1,y_0) > max(1, y_0) = 1$). Но это противоречит условию 4, что $h(x,y) \neq 1 \; \forall x, y \in [0, 1]$ ввиду монотонного неубывания $h(x,y)$ по обоим аргументам и $h(1,1) = 1$.  Если бы $h(1, y_0) > 1$, то и $h(1,1)$ должна быть больше 1, что противоречит условию.

Аналогично, если  $x_0 < 1$ и $y_0 = 1$.

Если же $0 < x_0 < 1$ и $0 < y_0 < 1$, и  $h(x_0, y_0) > max(x_0, y_0)$, то можно построить последовательность точек, приближающихся к (1, 1),  где значение $h$ будет продолжать расти, что приведет к противоречию с условием $h(1, 1) = 1$ из-за монотонности $h$.

Следовательно изначальное предположение неверно и $h(x,y)\equiv max(x,y).\blacksquare$


\end{document}