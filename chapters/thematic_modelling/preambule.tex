% Русский язык
\renewcommand{\familydefault}{\sfdefault}% шрифт
\usepackage[T2A]{fontenc} % кодировка
\usepackage[utf8]{inputenc} % кодировка исходного кода
\usepackage[english,russian]{babel} % локализация и переносы
% Отступы
\usepackage[left=2cm,right=2cm,top=2cm,bottom=3cm,bindingoffset=0cm]{geometry}
\usepackage{indentfirst}
% Вставка картинок
\usepackage{graphicx}
\graphicspath{}
\DeclareGraphicsExtensions{.pdf,.png,.jpg,.jpeg}

% Таблицы
\usepackage[table,xcdraw]{xcolor}
\usepackage{booktabs}

% Ссылки
\usepackage{hyperref}
% \bibliographystyle{unsrt}
% Для ссылки на библиографию references.bib нужно добавить строчку в конец файла
% \bibliography{references}
% Математика
\usepackage{physics}
\usepackage{amssymb, amsmath}

% Для статьи
\usepackage{abstract}
% \usepackage{lipsum}
% \newcommand\abstractname{Abstract}

% Для tikz
\usepackage{pgfplots}
\usepackage{physics}
% \usepackage{tikz}
\pgfplotsset{compat=1.15}
\usepackage{mathrsfs}
\usetikzlibrary{arrows}

% Мои команды
\DeclareMathOperator*{\sign}{sign}
\DeclareMathOperator*{\Real}{Re}
\DeclareMathOperator*{\Imag}{Im}

% Скобки через \br
\usepackage{mathtools}
\DeclarePairedDelimiter\autobracket{(}{)}
\newcommand{\br}[1]{\autobracket*{#1}}

% Окружение для многострочных уравнений
\usepackage{empheq}
\newenvironment{eqw}{\begin{equation} \begin{aligned}}{\end{aligned}    \end{equation}}
\newenvironment{eqw*}{\begin{equation*} \begin{aligned}}{\end{aligned}    \end{equation*}}

